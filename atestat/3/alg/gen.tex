\vspace{1cm}
\subsubsection{Generarea mutărilor legale}
\vspace{1cm}

Generarea mutărilor este o parte esențială, deci trebuie să fie cât mai rapidă,
astfel pentru fiecare tip de piesă și fiecare pătrat se precalculează un tabel
care conține un \textit{BitBoard} cu valori de 1 în pătratele care sunt atacate
de fiecare piesă pe pătratul respectiv.
\vspace{0.2cm}

Această metodă funcționează pentru pioni, cai și regi dar nu și pentru ture, nebuni și
regine deoarece ele pot fi blocate de alte piese. Se observă că regina funcționează ca
o tură combinată cu un nebun, deci nu avem nevoie să generăm nimic în plus.
\vspace{0.2cm}

Pentru ture și nebuni se folosește o metodă numită \textit{Magic BitBoards} care constă în generarea
unui \textit{BitBoard} pentru fiecare aranjament piese care pot bloca piesa pentru care
generăm. La prima vedere pare o metodă ineficientă, dar tura atacă 14 pătrate pe o tablă goală
iar nebunul maxim 13. Aceste numere se pot reduce la 12, respectiv 9 observând că pătratele
de la marginea tablei nu schimbă \textit{BitBoardul} generat indiferent de starea de ocupație,
ceea ce înseamnă că pentru fiecare pătrat trebuie să generăm ${2^{12} + 2^9 = 4608}$ de numere,
un număr foarte mic.

Rezultatele generării sunt memorate într-o structură numită \textit{Precalc}:
\begin{lstlisting}[language=RustHtml]
pub struct Precalc {
    pub bishop_magic: Box<[[BitBoard; 512]; 64]>,
    pub rook_magic: Box<[[BitBoard; 4096]; 64]>,
    pub bishop: Box<[(BitBoard, u32); 64]>,
    pub rook: Box<[(BitBoard, u32); 64]>,
    pub pawns: Box<[[BitBoard; 2]; 64]>,
    pub knight: Box<[BitBoard; 64]>,
    pub king: Box<[BitBoard; 64]>,
}
\end{lstlisting}
