\newpage
\begin{center}
	\section{Limbaje folosite}
\end{center}
\vspace{2cm}

\begin{itemize}
	\item \textbf{HTML} sau HyperText Markup Language, este un limbaj de marcă utilizat
	      pentru crearea și structurarea conținutului unei pagini web. Prin intermediul
	      unui set de etichete și atributelor, HTML definește elementele de bază ale unei
	      pagini web, cum ar fi titlurile, paragrafele, imagini și link-uri. Acesta permite
	      dezvoltatorilor să organizeze și să formateze conținutul într-un mod structurat
	      și semnificativ, esențial pentru crearea paginilor web ușor de înțeles și de navigat.
	\item \textbf{CSS}, sau Cascading Style Sheets, este un limbaj de stilizare folosit
	      pentru a formata și a stiliza aspectul vizual al unei pagini web. Prin definirea
	      regulilor de stil, precum culoarea, fontul, dimensiunea și alinierea elementelor
	      HTML, CSS permite dezvoltatorilor să creeze aspectul și simțul dorit pentru
	      site-uri web. Prin separarea conținutului și a prezentării, CSS îmbunătățește
	      flexibilitatea și consistența în designul web, facilitând personalizarea și
	      gestionarea aspectului unei pagini web în mod eficient.
	\item \textbf{TypeScript} TypeScript este un limbaj de programare care extinde limbajul
	      JavaScript, adăugând tipurile statice la limbaj. Acesta oferă programatorilor
	      posibilitatea de a defini tipuri pentru variabile, argumente de funcții și
	      valori de returnare, permițând astfel detectarea erorilor în timpul dezvoltării
	      și îmbunătățirea robusteții codului. Prin adăugarea tipurilor, TypeScript
	      facilitează înțelegerea și menținerea codului, crește productivitatea și
	      reduce numărul de erori în timpul executării.

	\item \textbf{Rust} este un limbaj de programare modern și eficient, proiectat
	      pentru a oferi performanță și siguranță la nivel de sistem, fără a compromite
	      abordarea ergonomică și ușurința de utilizare. Prin intermediul sistemului
	      său de tipuri puternic, Rust permite scrierea de cod care este sigur din
	      punct de vedere al memoriei, fără a fi necesare tehnici complexe de
	      gestionare a resurselor.

	\item \textbf{WebAssembly (WASM)} WASM, sau WebAssembly, este un format
	      binar și un limbaj de reprezentare a codului portabil, proiectat pentru a
	      fi executat în medii web. Acesta oferă o alternativă eficientă și sigură la
	      JavaScript pentru executarea codului în browser, permițând compilarea codului
	      într-un format compact și performant. Datorită performanței ridicate și a
	      interoperabilității cu limbaje precum C/C++ și Rust, WASM devine o
	      alegere atractivă pentru crearea aplicațiilor web complexe și interactive,
	      cum ar fi jocurile, aplicațiile de editare foto/video și multe altele.
\end{itemize}
